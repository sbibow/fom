\newpage
\section{Microservice Architekturen} \label{msa}

% Im Kapitel \ref{msa} wird zunächst die MSA erklärt und in die allgemeinen
% Software Architektur eingeordnet. Auch wiord in diesem Kapitel die Monolithische
% Architektur vorgestellt und mit der MSA kontrastiert.

MSA ist ein Architekturmuster, das eine Applikation in viele Services aufteilt.
Die einzelnen Services sind lose miteinander gekoppelt, wodurch unabhängige
Entwicklung und Bereitstellung des Services ermöglicht werden.

\subsection{Definition von Microservice Architekturen}

Um eine schlüssige Definition von MSA zu geben, stützt sich dieser Arbeit auf
die Ansichten von Martin Fowler. Fowler ist ein Autor mehrerer Bücher zum
Thema Softwarearchitektur\footnote{TODO: Referenzen einfügen}.

Auf seinem Blog nennt Fowler folgende Eigenschaften von
MSA\footnote{https://www.martinfowler.com/articles/microservices.html}:

\begin{enumerate}
  \item Komponenten als Services
  \item Organisation um Unternehmensfunktionen
  \item Produkte, nicht Projekte
  \item Schlaue Endpunkte und dumme Verbindungen
  \item Dezentralisierte Steuerung
  \item Dezentralisierte Datenverwaltung
  \item Infrastrukturautomatisierung
  \item Design für Fehlerresistenz
  \item Evolutionäres Design
\end{enumerate}

\subsubsection{Komponenten als Services}

Fowler beschreibt Komponenten als eine Softwareeinheit, die unabhängig von
anderen Komponenten ausgetauscht und geupgraded werden kann. Traditionell werden
diese Komponenten als Bibliothek bereitgestellt. Laut Fowler sind Bibliotheken
Komponenten, die im Programm gelinkt sind und über in-memory Funktionen
aufgerufen werden. Services hingegen sind Komponenten, die außerhalb des
Prozesses laufen und über Mechanismen wie Web Service Aufrufe oder \ac{RPC}
aufgerufen werden.

Fowler nennt die unabhängige Deploybarkeit als zentralen Vorteil von Services
gegenüber Bibliotheken. Um eine Bibliothek zu aktualisieren muss der komplette
Hauptprozess inklusive anderer Bibliotheken neu bereitgestellt werden. Bei
Services kann jeder Service unabhängig von den anderen aktualisiert werden -
vorrausgesetzt die \ac{API} verändert sich nicht.

Natürlich gibt es auch Nachteile von Services gegenüber Bibliotheken. Laut
Fowler sind RPC teurer als Zugriffe über den geteilten Speicher. Ebenfalls ist
die Migration von Funktionen von einem Service zum anderen schwieriger.
