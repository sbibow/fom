\newpage
\section{Einleitung} \label{Einleitung}
Ein wichtiger Erfolgfaktor für Unternehmen ist es aktuelle Trends zu erkennen
und ihre Strategie entsprechend anzupassen. 


\subsection{Forschungsfrage}
Wie können Unternehmen Hypes von erfolgsversprechenden Trends - vor allem im
Bereich der IT-Infrastruktur - unterscheiden?

\subsection{Aufbau der Arbeit}

Im Kapitel \ref{infrastruktur} wird dargestellt, welche IT Komponenten und
Systeme in der IT-Infrastruktur vorhanden sind. Hier wird ein besonderer Fokus
auf Cloud Technologien gelegt.

Im Kapitel \ref{methodik} wird eine Möglichkeit vorgestellt, wie man
IT-Infrastruktur-Trends bewerten kann.

Im Kapitel \ref{hybrid_cloud} wird dieses Verfahren auf Trends im Hybrid
Cloud Umfeld angewandt, um einen Hype zu erkennen und einen etablierten Trend
darzustellen.

Zuletzt werden die Ergebnisse dieser Semesterarbeit im Kapitel \ref{fazit}
zusammengefasst.

\subsection{Unterschied zwischen Trend und Hype}

In dieser Arbeit wird zwischen Trend und Hype unterschieden. Ein Trend
bezeichnet eine neue Technologie oder Arbeitsweise, die vielen Unternehmen einen
Wettberwerbsvorteil verschaffen - oder notwendig sind, um zumindest keinen
Wettbewerbsnachteil zu haben.

Ein Hype hingegen ist eine neue Technologie oder Arbeitsweise, die eine kurze
Zeit sehr viel Aufmerksamkeit durch Unternehmen und Gesellschaft bekommt, aber
wenig oder keinen Mehrwert für Unternehmen bietet. Der Hype stribt nach kurzer
Zeit aus (siehe auch Abbildung \ref{fig:trend_vs_hype}).

\begin{figure}[H]
\caption{Trend vs. Hype}\label{fig:trend_vs_hype}
\begin{tikzpicture}[xscale=10,yscale=5]
% Achsen
\draw[<->] 
     (1,0)
  -- (0.5, 0) node[below]{Zeit} 
  -- (0,0) 
  -- (0,0.5) node[left]{Erwarteter Erfolg}
  -- (0,1);

%hype
\draw[domain=0:0.3778,samples=100] plot (\x,{0.95*2^(-(11.9*(\x-0.3))^2)+0.04});
\draw[domain=0.3778:1,samples=100,orange] plot (\x,{0.95*2^(-(11.9*(\x-0.3))^2)+0.04});
\draw (0.6,0.05) node[above]{Hype};
%trend
\draw[blue]
  (0.3778,0.564166) to [out=-82.88,in=-180] (0.58,0.37) to [out=0,in=-150] (1,0.7);
\draw (0.6,0.5) node[below]{Trend};

\end{tikzpicture}
\\
Quelle: Eigene Darstellung basierend auf Gartner Hype-Cycle
\end{figure}
