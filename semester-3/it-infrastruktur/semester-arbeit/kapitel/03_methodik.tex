\newpage
\section{Methodik} \label{methodik}

% Es gibt viele Probleme (und Lösungen) im Sicherheitsumfeld. Damit Unternehmen
% diese priorisieren können, müssen diese Probleme nach einem festen Schema
% bewertet werden. Dieses Kapitel baut auf anderen Arbeiten auf, um ein eigenes
% Bewertungsschema aufzustellen, das in späten Kapiteln genutzt wird, um die
% wichtigsten Sicherheitsdefizite zu identifizieren und zu beheben.

% %- Nutzen von Attack Matrix für Kubernetes, um Lücken zu finden
% %- Nutzen von modifiziertem CVSS für Bewertung von Lücken

% \subsection{Evaluation bisheriger Bewertungssysteme}

% \subsubsection{Common Vulnerability Scoring System}

% Das Common Vulnerability Scoring System (CVSS) ist ein Industriestandard zur
% Bewertung von Softwareschwachstellen. Es gibt einen einzigen Wert zwischen 0
% und 10 zurück.

In dieser Arbeit werden viele Sicherheitsschwachstellen vorgestellt; nicht alle
sind gleich verherend in ihrer tatsächlichen Auswirkung auf die Sicherheit der
Plattform.

Zunächst werden in Kapitel \ref{sicherheitsrisiken} potentielle Angreifer
beschrieben und wie sie die Sicherheit der Plattform schwächen können. Diese
Taktiken werden im selben Kapitel zu konkreten Sicherheitsdefiziten
ausformuliert.





