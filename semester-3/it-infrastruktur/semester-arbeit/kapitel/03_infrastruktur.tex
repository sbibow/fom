\newpage
\section{Historie von Hybrid Cloud} \label{infrastruktur}

\subsection{Arten des Cloud Computing}

Cloud Computing bezeichnet ein Modell, durch das Rechnerressourcen (Netzwerk,
Rechenkapazität, Speicherplatz, Applikationen und Services) über ein
Netzwerzugriff bereitgestellt werden. Das \ac{NIST} nennt drei Service Modelle
und vier Deployment Modelle. Die Service Modelle sind: \ac{SaaS}, \ac{PaaS} und
\ac{IaaS}. Die Deployment Modelle sind: Private Cloud, Community Cloud, Public
Cloud und Hybrid Cloud.

\subsection{Die Stärken von Private und Community Clouds}
% - Im Vergleich zu Public Cloud - nicht im Vergleich zu traditionellen Rechenzentrumsbetrieb
% - Mehr Kontrolle über Daten (eg. für Regulatoren, Zertifizierungen)
% - Kann günstiger sein, transparenter, was später abgerechnet wird
% - Mehr flexibilität in Architektur
% - Deutlich, wer für was verantwortlich ist
% - Einfach, wenn schon eigene Infrastruktur vorhanden
% - 


