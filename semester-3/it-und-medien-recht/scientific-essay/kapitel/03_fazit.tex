\newpage
% Abschluss:
%   - Zusammenfassung der eingebrachten Perspektiven
%   - Fazit

\section{Fazit} \label{fazit}

Google Street View sollte nicht erweitert werden. Es gibt einige Vorteile durch
Google Street View, doch ist mir die Privatsphäre wichtiger. Durch das
Anfertigen dieser Arbeit wurde mir deutlich, wie viele Informationen alleine
durch die Bilder ersichtlich sind.

Das aktuelle Recht ist nicht darauf ausgelegt, mit dem Internetgiganten Google
klarzukommen. Google stellt ein Komglomerat an Diensten dar; bestehend aus
Suchmaschine, Mail-Anbieter(GMail), Werbeagentur (Google Ads und AdSense),
Entertainment-Provider (YouTube), Betriebssystementwickler (Android),
Sicheheitstechnik-Vermarkter (Nest), Internetanbieter (Google Fiber) und
Massendatenhalter in Form von Google Street View.

Die Korrelationen, die Forscher:innen mit den Google Street View Bildern
anfertigen konnten, sind beängstigend. Es lässt sich nur vermuten, was Google
mit dem Verbund aller Daten erreicht. Diese Informationen können genutzt
werden, um Werbung noch besser und gezielter zu vermitteln oder sogar die
politische Meinung zum Beispiel durch die Vorschläge bei YouTube zu prägen.

Das deutsche Recht ist mit einem solchen Anbieter überfordert und hält
fälschlicherweise an Gesetzen fest, die in einer Zeit geschrieben wurden, in
denen eine Massendatenerhebung und -haltung nicht möglich war.

Google Street View ist nur ein weiteres Produkt von Google, das zeigt, wie
durchsichtig der Mensch im digitalen Zeitalter geworden ist.
