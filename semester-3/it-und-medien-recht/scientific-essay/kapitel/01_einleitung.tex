\newpage
% Einleitung:
%   - Bezug zum Modul oder aktueller Thematiks
%   - Fragestellung bzw. Problem

\section{Einleitung} \label{Einleitung}
Diese Arbeit beschäftigt sich mit Google Street View und den Implikationen für
die Privatsphäre von Personen.

Google Street View ist ein Dienst, der es möglich ist mit einem
Internet-Browser oder der mobilen Google Maps App 360\degree-Ansichten aus der
Straßenperspektive einzusehen.

Google Street View wurde erstmals 2007 in einigen Städten in den USA
eingeführt\footcite{website:heise:google-street-view-einfuehrung-usa} und 2010
in 20 Deutschen Städten verfügbar
gemacht\footcite{website:sueddeutsche:20-deutsche-stadte-online}. Seitdem gab es
keine Erweiterung von Google Street View in Deutschland.

\subsection{These}

Google Street View sollte in Deutschland erweitert und aktualisiert werden.

\subsection{Bezug zum Modul}

In den Vorlesungen wurde deutlich, dass persönliche Daten zu schützen sind.
Andernfalls werden diese Daten genutzt, um gezieltes Marketing zu schalten,
Personen zu überwachen oder bestimmte Personen aus Versicherungen oder Krediten
auszuschließen.

Daher gibt es im IT- \& Medienrecht viele Gesetze und Rechtssprechungen, die
personenbezogene Daten vor privaten Unternehmen schützen sollen.

In dieser Arbeit wird unter anderem gezeigt, welche Daten aus Google Street View
ausgelesen werden können und welche Urteile es gegen bzw. für Goggle in Bezug
auf Street View gibt.
